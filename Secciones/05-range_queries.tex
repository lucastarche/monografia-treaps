\section{Range Queries}

Muchos problemas de estructuras de datos consisten en calcular una función acumulada sobre un rango de un arreglo, como la suma de elementos en el intervalo \([l, r]\).
Es decir, si nuestro arreglo es \(a_1, \dots, a_n\), queremos calcular 

\[
f(l, r) = a_l + a_{l + 1} + \dots + a_r = \sum_{i = l}^r{a_i}
\]

En general, esto vale para cualquier \textbf{monoide}.
Decimos que una operación binaria \(\oplus\) sobre un conjunto \(A\) es un monoide si es:

\begin{itemize}
\item Cerrada: si \(a, b \in A\), \(a \oplus b \in A\)
\item Asociativa: \((a \oplus b) \oplus c = a \oplus (b \oplus c)\) para todos \(a, b, c \in A\)
\item Elemento neutro: existe un (único) elemento \(e \in X\) tal que \(e \oplus a = a \oplus e = a\) para todo \(a \in A\)
\end{itemize}

Hay varios ejemplos comunes de monoides, incluyendo la suma, el producto, máximo y mínimo, entre otros. 
De hecho, ambas definiciones para el tamaño de un nodo eran monoides.

De manera similar al tamaño, definimos para cada nodo \(x\) la función acumulada en su subárbol como

\[
f(x) = f(x_L) \oplus x_a \oplus f(x_R)
\]

donde \(f\) de un nodo nulo es la identidad del monoide \(\oplus\), y \(x_a \in A\) es algún valor asociado a \(x\).
En el caso del tamaño, \(x_a\) era la frecuencia del valor.

Para usar un Treap para resolver esta clase de problemas, insertamos las claves \(1, \dotsm, n\), cada una con su valor original. 
De esta forma, computar \(f(l, r)\) se reduce a dos llamados de Split(), para obtener los elementos con clave en \([l, r]\), y la respuesta es la función acumulada de la raíz:

\vbox{
\begin{minted}{c++}
int query(Treap &root, int l, int r) {
    Treap left, middle, right;
    tie(left, right) = split(root, r);
    tie(left, middle) = split(left, l - 1);
    
    int ans = f(middle);
    root = merge(merge(left, middle), right);
    return ans;
}
\end{minted}
}

\subsection{CSES - Dynamic Range Sum Queries}
\label{sec:dynamic-range-sum-queries}
\problema{
\textbf{Dynamic Range Sum Queries:}

Tenemos una lista con números \(a_1, \dots, a_n\), y queremos responder \(q\) consultas del tipo: 
\begin{enumerate}
    \item Cambiar \(a_k\) por \(x\)
    \item Computar la suma en \([l, r]\)
\end{enumerate}

\url{https://cses.fi/problemset/task/1648}
}

Como la suma es un monoide (cuya identidad es cero), ya sabemos computar sumas en rango de manera eficiente.
Para cambiar el valor en una posición, particionamos el árbol en tres con respecto a \(k\), actualizamos el valor, y juntamos todo de nuevo.

\vbox{
\begin{minted}{c++}
void update(Treap &root, int k, int x) {
    Treap left, middle, right;
    tie(left, right) = split(root, k);
    tie(left, middle) = split(left, k - 1);

    middle->value = x;
    root = merge(merge(left, middle), right);
}
\end{minted}
}

Tanto Update() como Query() funcionan en \(\mathcal{O}(\lg n)\), entonces la complejidad de la solución es \(\mathcal{O}(n + q \lg n)\). 

\subsection{CSES - Range Updates and Sums}

\problema{
\textbf{Range Updates and Sums:} 

Tenemos una lista de números \(a_1, \dots, a_n\), y queremos responder \(q\) consultas del tipo:
\begin{enumerate}
\item Sumarle \(x\) a cada elemento en \([l, r]\)
\item Reemplazar por \(x\) cada elemento en \([l, r]\)
\item Computar la suma en \([l, r]\)
\end{enumerate}

\url{https://cses.fi/problemset/task/1735/}
}

Notar que, si directamente aplicáramos las operaciones de tipos 1. y 2. apenas llegan,
la cantidad de posiciones a cambiar es \(\mathcal{O}(n \cdot q)\), que son demasiadas operaciones para las cotas del problema.
En su lugar, vamos a marcar que tenemos que actualizar un subárbol, e ir propagando las actualizaciones cuando visitamos un nodo (un truco conocido como \textit{Lazy Propagation}).

Para poder hacer Lazy Propagation, necesitamos:
\begin{itemize}
    \item Una forma de computar la función acumulada de un nodo, incluso si tiene actualizaciones pendientes
    \item Una forma de componer actualizaciones
    \item Una forma de aplicar las actualizaciones pendientes a un nodo, y propagarlas a sus hijos
\end{itemize}

En cada nodo, vamos a guardar tres valores adicionales, que representan las actualizaciones que tiene pendientes:
un entero diciendo cuánto hay que añadirle al subárbol (\texttt{lazyAdd}), un entero diciendo cuánto hay que asignarle a cada nodo del subárbol (\texttt{lazySet}), y un booleano que dice si hay que hacer una asignación o no. 
Este último parámetro es necesario, ya que no existe una ``identidad'' para asignar valores.

Para computar la función acumulada de un nodo, tenemos dos casos, dependiendo de si hay o no una actualización de reemplazo pendiente.

Si tenemos que realizar una asignación:
\[
f(x) = |x| \cdot (x \member{lazyAdd} + x \member{lazySet})
\]

Caso contrario, es igual a:
\[
f(x) = f(x_L) + f(x_R) + x_a + |x| \cdot x \member{lazyAdd}
\]

Para componer con una actualización de suma, le sumamos la cantidad a \texttt{lazyAdd}.
Por otro lado, si queremos componer con un reemplazo, tenemos que reemplazar \texttt{lazyAdd} por cero, y asignarle a \texttt{lazySet} el nuevo valor.

Finalmente, para propagar actualizaciones, el valor del nodo es \(x \member{lazySet} + x \member{lazyAdd} \) si hay que asignar, y \(x_a + x \member{lazyAdd}\) si no.

De manera similar al problema anterior, para hacer una actualización hay que buscar el rango afectado con Split(), y componer con la nueva actualización. 
Solo es necesario propagar las actualizaciones necesarias cuando visitamos un nodo en Split() o Merge():

\begin{minted}{c++}
pair<Treap, Treap> split(Treap root, tint k) {
  if (root == nullptr) return {nullptr, nullptr};

  propagate(root);
  if (root->key <= k) // ...
}

Treap merge(Treap left, Treap right) {
  if (left == nullptr) return right;
  if (right == nullptr) return left;

  propagate(left), propagate(right);
  if (left->priority > right->priority) // ...
}
\end{minted}

Como sólo hacemos una cantidad constante de operaciones adicionales en Split() y Merge(), la complejidad de este problema también es \(\mathcal{O}(n + q \lg n)\).
