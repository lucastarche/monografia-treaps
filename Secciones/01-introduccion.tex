\section{Introducción}

En la literatura, muchos problemas pueden ser resueltos adaptando un árbol de búsqueda binaria balanceado, como es el caso de un Árbol AVL o Árbol Red-Black.
Sin embargo, la implementación de ambas estructuras es larga y con varios casos, lo que no es óptimo para competencias.
En su lugar, presentamos el Treap o Árbol de Búsqueda Binaria Aleatorio, cuya implementación es simple y concisa,
y mantiene el tiempo esperado con una muy alta probabilidad.

Vamos a usar esta estructura para resolver varios problemas de competencia,
así como mostrar técnicas y trucos para aumentar árboles binarios,
incluyendo "Treap Implícito" para manipulado de secuencias, y cómo añadirle persistencia a la estructura.