\section{Treap Implícito}

Vamos a ver un truco particularmente útil para manipular secuencias y cadenas,
conocido como ``Treap Implícito'' o ``Treap con Claves Implícitas''. 
A diferencia de los problemas que vimos hasta ahora, los tres problemas a continuación no
pueden resolverse usando alguna variante de Segment Tree.

Un patrón clásico en problemas es añadir consultas de cortar/pegar y/o invertir rangos,
para evitar usar un Segment Tree, como es el caso de \nameref{sec:reversals-and-sums}.
En su lugar, debe aumentarse un ABB autobalanceado, siendo Treap y Splay Tree
los dos más populares para competencias.

\subsection{CSES - Cut and Paste}

\problema{
\textbf{Cut and Paste:} 

Tenemos una cadena \(S = s_1 \dots s_n\), y vamos a cortar una subcadena \(s_l \dots s_r\) y pegarla al final \(q\) veces.

Imprimir la cadena después de todas las operaciones.

\url{https://cses.fi/problemset/task/2072}
}

Notar que, luego de una operación, pasamos de la cadena \(S = s_1 \dots s_n\) a la cadena \(S' = s_1 \dots s_{l - 1} s_{r + 1} \dots s_n s_l \dots s_r\).
Es decir, partimos \(S\) en tres porciones, y las pegamos en una nueva cadena. 
Uno querría insertar los pares índice, carácter en un Treap, de manera similar que para sumas, 
y usar Split() y Merge() para cortar y pegar, respectivamente.

Lamentablemente, esto no funciona, porque luego de una operación, el árbol podría dejar de tener la propiedad de búsqueda binaria, así como que las claves ya no representen los índices:

\begin{center}
\begin{tikzpicture}[
sibling distance=0.5in,
every node/.style={
    circle split,
    draw
}
]

\node { R \nodepart{lower} 2 }
    child { node { T \nodepart{lower} 1 } }
    child {
        node { A \nodepart{lower} 4 }
        child { node { E \nodepart{lower} 3 } }
        child { node { P \nodepart{lower} 5 } }
    }
;
\end{tikzpicture}
\begin{tikzpicture}[
sibling distance=0.5in,
every node/.style={
    circle split,
    draw
}
]

\node { R \nodepart{lower} 2 }
    child { node { T \nodepart{lower} 1 } }
    child {
        node { E \nodepart{lower} 3 }
        child { node { P \nodepart{lower} 5 } }
        child { node { A \nodepart{lower} 4 } }
    }
;
\end{tikzpicture}
\captionof{figure}{Intentando hacer ``TREAP'' \(\to\) ``TRPEA''}
\end{center}

Queremos alguna forma de actualizar todos los índices cuando cortamos el árbol,
pero sin tener que recorrer los nodos uno por uno.
Para evitar esto, la idea va a ser no guardarnos las claves en el árbol,
sino deducirlas en el momento con la estructura del árbol.

Definimos la \textit{clave implícita} de un nodo como su posición en el inorder de recorrer el árbol desde la raíz.
Aunque no estén almacenados explícitamente en cada nodo, los podemos obtener a medida que vamos recorriendo 
el árbol para hacer Split(). 
Hay que modificar ligeramente Split() para que funcione con claves implícitas:

\vbox{
\begin{minted}{c++}
pair<Treap, Treap> split(Treap root, int k) {
 if (root == nullptr) return {nullptr, nullptr};
  
 int key = size(root->left) + 1;
 Treap left, right;
 if (key <= k) {
  tie(left, right) = split(root->right, k - key);
  root->right = left;
  evaluate(root);
  return {root, right};
 } else {
  tie(left, right) = split(root->left, k);
  root->left = right;
  evaluate(root);
  return {left, root};
 }
}
\end{minted}
}

Como los nodos no tienen claves, no hay que mantener su orden, entonces podemos pegar árboles que, conceptualmente, no están en órden. 
Esta propiedad hace que la implementación sea lo que habíamos pensado originalmente:

\vbox{
\begin{minted}{c++}
void update(Treap &root, int l, int r){
  Treap left, middle, right;
  tie(left, right) = split(root, r);
  tie(left, middle) = split(left, l - 1);
  root = merge(merge(left, right), middle);
}
\end{minted}
}

Finalmente, imprimir el resultado es concatenar los caracteres de los nodos en recorrido inorder, y la complejidad total es \(\mathcal{O}(n + q \lg n)\).

\subsection{CSES - Substring Reversals}
\problema{
\textbf{Substring Reversals:} 

Tenemos una cadena \(S = s_1 \dots s_n\), y vamos a dar vuelta subcadenas \(s_l \dots s_r\) un total de \(q\) veces.

Imprimir la cadena después de todas las operaciones.

\url{https://cses.fi/problemset/task/2073}
}

Vamos a armar un Treap Implícito con los caracteres, de la misma forma que en el problema anterior. 
Notar que, si no fuera implícito, podría romperse el orden de las claves luego de una actualización.

Recordemos que, para invertir un árbol binario, intercambiamos los hijos izquierdo y derecho de la raíz, y procedemos recursivamente.
Además, invertir dos veces un árbol nos devuelve el árbol original.
Esto nos da la idea de usar Lazy Propagation para procesar rápido las actualizaciones, sin tener que recorrer todos los nodos afectados.

Sólo hace falta guardarse un booleano adicional por nodo,
que nos indica si hay una actualización pendiente o no.
El resultado es sorprendente porque, si omitimos imprimir la respuesta, podemos dar vuelta una subcadena en \(\mathcal{O}(\lg n)\).
Obtener la cadena final es cuestión de recorrer el árbol en recorrido inorder, propagando actualizaciones a medida que visitamos los nodos.

\begin{center}
\begin{tikzpicture}[
sibling distance=0.5in,
every node/.style={
    circle,
    draw
}
]
\node[fill=lightgray] { R \nodepart{lower} 2 }
    child { node { T } }
    child {
        node { A }
        child { node { E } }
        child { node { P  } }
    }
;
\end{tikzpicture}
\begin{tikzpicture}[
sibling distance=0.5in,
every node/.style={
    circle,
    draw
}
]
\node { R \nodepart{lower} 2 }
    child {
        node[fill=lightgray] { A }
        child { node { E } }
        child { node { P  } }
    }
    child { node[fill=lightgray] { T } }
;
\end{tikzpicture}
\begin{tikzpicture}[
sibling distance=0.5in,
every node/.style={
    circle,
    draw
}
]
\node { R \nodepart{lower} 2 }
    child {
        node { A }
        child { node[fill=lightgray] { P  } }
        child { node[fill=lightgray] { E } }
    }
    child { node { T } }
;
\end{tikzpicture}
\captionof{figure}{Dar vuelta ``TREAP'' \(\to\) ``PAERT''}
\end{center}

\subsection{CSES - Reversals and Sums}
\label{sec:reversals-and-sums}

\problema{
\textbf{Reversals and Sums:} 

Tenemos una lista de números \(a_1 \dots a_n\), y queremos responder \(q\) consultas del tipo:

\begin{enumerate}
\item Dar vuelta el rango \([l, r]\)
\item Computar la suma en el rango \([l, r]\)
\end{enumerate}

\url{https://cses.fi/problemset/task/2074}
}

La solución a este problema es casi inmediata, sólo tenemos que combinar la respuesta al problema anterior con la solución a \nameref{sec:dynamic-range-sum-queries}.
Esto es fácil de hacer, ya que no cambia la suma de un subárbol al invertirlo.

De este problema, podemos notar lo siguiente:
\begin{itemize}
\item Podríamos haber usado claves implícitas para resolver los problemas de Range Queries.
\item Es fácil añadirle consultas de cortar/pegar o invertir rangos a un problema de Range Queries, si tenemos una solución con Treap.
\end{itemize}

De hecho, podríamos haber resuelto el problema ``Range Updates, Sums and Reversals'' de la misma forma.